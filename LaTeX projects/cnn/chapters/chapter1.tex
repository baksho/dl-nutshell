\chapter{Introduction}\label{chp:1}
Convolutional Neural Networks (CNNs) are a type of deep learning model \textbf{designed to recognize patterns in visual data}, such as images and videos. Unlike traditional neural networks, CNNs exploit the spatial and hierarchical patterns present in the data to efficiently learn features. CNNs consist of layers that perform convolutions, mathematical operations that filter and extract features such as edges, textures, and shapes from images. These networks typically have three main types of layers: \textbf{convolutional layers}, \textbf{pooling layers} (to reduce the image size), and \textbf{fully connected layers} (for final classification). CNNs are particularly powerful for tasks like image recognition, object detection, and facial recognition, as they automatically learn relevant features from the raw input data.

% Adding a figure
\begin{figure}[h!]
    \centering
    \includegraphics[width=0.9\textwidth]{images/figure1.png}
    \caption{A typical Convolutional Neural Network}
    \label{fig:1}
\end{figure}

\section{Key Characteristics of CNNs}
\begin{itemize}
    \item \textbf{Convolutions}: A mathematical operation that extracts patterns such as edges and textures.
    \item \textbf{Hierarchical Feature Learning}: CNNs learn low-level features (e.g., edges) in initial layers and complex patterns (e.g., objects) in deeper layers.
    \item \textbf{Parameter Sharing}: Convolutions reduce the number of parameters, making them computationally efficient.
    \item \textbf{Translation Invariance}: Recognize patterns regardless of their position in the input.
\end{itemize}
For instance, CNNs are capable of classifying an image of a cat even if the cat appears in different parts of the image.

\section{Why CNNs are important in Machine Learing}
A \textbf{Convolutional Neural Network (ConvNet/CNN)} can take in an input image, assign importance (learnable weights and biases) to various aspects/objects in the image, and be able to differentiate one from the other. \textbf{The pre-processing required in a ConvNet is much lower as compared to other classification algorithms}. While in primitive methods, filters are hand-engineered; with enough training, ConvNets have the ability to learn these filters/characteristics.

\setlength{\parindent}{0pt}CNNs have revolutionized the field of computer vision and beyond by enabling machines to:
\begin{itemize}
    \item \textbf{Understand visual content}: Perform image classification, object detection, and segmentation.
    \item \textbf{Process structured data efficiently}: Learn features directly from raw input, reducing the need for manual feature engineering.
    \item \textbf{Achieve human-level accuracy}: Outperform traditional methods in tasks like face recognition, medical imaging, and natural language processing.
\end{itemize}

The architecture of a ConvNet is analogous to that of the connectivity pattern of Neurons in the Human Brain and was inspired by the organization of the Visual Cortex. Individual neurons respond to stimuli only in a restricted region of the visual field known as the Receptive Field. A collection of such fields overlaps to cover the entire visual area.

\section{Why ConvNets over Feed-Forward Neural Nets?}

An image is nothing but a matrix of pixel values. So why not just flatten the image (e.g. 3x3 image matrix into a 9x1 vector) and feed it to a Multi-Level Perceptron for classification purposes?

For extremely basic binary images, this method might show an average precision score while predicting classes but would fail miserably for complex images having pixel dependencies.

\newpage
\begin{figure}[h!]
    \centering
    \includegraphics[width=0.4\textwidth]{images/figure2.png}
    \caption{Flattening of a 3x3 image matrix into a 9x1 vector}
    \label{fig:2}
\end{figure}

A ConvNet can \textbf{successfully capture the Spatial and Temporal dependencies} in an image through the application of relevant filters. The architecture performs a better fitting to the image dataset due to the reduction in the number of parameters involved and the reusability of weights.

\fbox{%
    \parbox{\textwidth}{%
        \textbf{Spatial Dependency} means a pixel's value is influenced by nearby pixel's value in image. This is because generally they all belong to same color because they are from same object. \textbf{Temporal dependency} comes in videos. When a frame changes to next, if there is not a lot of movement in objects, pixel's values remain same.
    }%
}
\begin{table}[h!]
\centering
\begin{tabular}{|p{0.2\linewidth}|p{0.35\linewidth}|p{0.35\linewidth}|} 
  \hline
  \rowcolor{lightgray} \textbf{Aspect} & \textbf{Feed – Forward Network (ANNs)} & \textbf{Convolutional Neural Network (CNNs)} \\ 
  \hline\hline
  \textbf{Input Representation} & Flat, requires feature extraction & Structured (e.g., images as grids of pixels) \\ 
  \hline
  \textbf{Architecture} & Fully connected layers & Convolutional + pooling layers + fully connected layers \\ 
  \hline
  \textbf{Parameter Count} & High (independent weights for each connection) & Low (shared weights through convolutions) \\
  \hline
  \textbf{Feature Learning} & Manual or less efficient & Automatic and hierarchical \\
  \hline
  \textbf{Applications} & Generic machine learning & Specialized for spatial and structured data \\
  \hline
\end{tabular}
\caption{Difference between ANNs and CNNs}
\label{table:1}
\end{table}

\section{Visual Representation of CNN workflow}

The below image illustrates a simple CNN architecture to classify handwritten digits from MNIST dataset. The digits are passed as input images of shape $28 \times 28$ to the network and output is 10 classes (from 0 to 9) to classify the image in one of them.


\newpage
\begin{figure}[h!]
    \centering
    \includegraphics[width=0.9\textwidth]{images/figure3.png}
    \caption{A CNN sequence to classify handwritten digits}
    \label{fig:3}
\end{figure}

The input image is passed through a sequence of layers (convolutions, pooling, and fully connected layers) to classify it into one of the 10 categories.


